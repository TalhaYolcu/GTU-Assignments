\documentclass[a4 paper]{article}
\usepackage[inner=2.0cm,outer=2.0cm,top=2.5cm,bottom=2.5cm]{geometry}
\usepackage{setspace}
\usepackage[rgb]{xcolor}
\usepackage{verbatim}
\usepackage{subcaption}
\usepackage{amsgen,amsmath,amstext,amsbsy,amsopn,tikz,amssymb}
\usepackage{fancyhdr}
\usepackage[colorlinks=true, urlcolor=blue,  linkcolor=blue, citecolor=blue]{hyperref}
\usepackage[colorinlistoftodos]{todonotes}
\usepackage{rotating}
\usepackage{booktabs}
\newcommand{\ra}[1]{\renewcommand{\arraystretch}{#1}}

\newtheorem{thm}{Theorem}[section]
\newtheorem{prop}[thm]{Proposition}
\newtheorem{lem}[thm]{Lemma}
\newtheorem{cor}[thm]{Corollary}
\newtheorem{defn}[thm]{Definition}
\newtheorem{rem}[thm]{Remark}
\numberwithin{equation}{section}

\newcommand{\homework}[6]{
   \pagestyle{myheadings}
   \thispagestyle{plain}
   \newpage
   \setcounter{page}{1}
   \noindent
   \begin{center}
   \framebox{
      \vbox{\vspace{2mm}
    \hbox to 6.28in { {\bf CSE 211:~Discrete Mathematics \hfill {\small (#2)}} }
       \vspace{6mm}
       \hbox to 6.28in { {\Large \hfill #1  \hfill} }
       \vspace{6mm}
       \hbox to 6.28in { {\it Name: {\rm #3} \hfill  {\rm #5} \hfill  {\rm #6}} \hfill}
       \hbox to 6.28in { {\it Student Number: #4  \hfill #6}}
      \vspace{2mm}}
   }
   \end{center}
   \markboth{#5 -- #1}{#5 -- #1}
   \vspace*{4mm}
}

\newcommand{\problem}[2]{~\\\fbox{\textbf{Problem #1}}\hfill (#2 points)\newline\newline}
\newcommand{\subproblem}[1]{~\newline\textbf{(#1)}}
\newcommand{\D}{\mathcal{D}}
\newcommand{\Hy}{\mathcal{H}}
\newcommand{\VS}{\textrm{VS}}
\newcommand{\solution}{~\newline\textbf{\textit{(Solution)}} }

\newcommand{\bbF}{\mathbb{F}}
\newcommand{\bbX}{\mathbb{X}}
\newcommand{\bI}{\mathbf{I}}
\newcommand{\bX}{\mathbf{X}}
\newcommand{\bY}{\mathbf{Y}}
\newcommand{\bepsilon}{\boldsymbol{\epsilon}}
\newcommand{\balpha}{\boldsymbol{\alpha}}
\newcommand{\bbeta}{\boldsymbol{\beta}}
\newcommand{\0}{\mathbf{0}}


\begin{document}
\homework{Homework \#3}{Due: 04/01/21}{Yakup Talha Yolcu}{1801042609}{}{}


\problem{1: Representing Graphs}{10}
\solution

\begin{figure*}[htb]
	\centering
	\includegraphics[scale=0.5]{matrix.png}
	\caption{Matrix}
	\label{fig:graph1}
\end{figure*}

In the vertical a-b-c-d-e I represented the going vertices. If any vertices have an edge to another point, I represented as 1, if not it is 0.



\newpage
\problem{2: Hamilton Circuits}{10+10+10=30}
 \newline

\begin{itemize}
\item According to Dirac Theorem, Let G be a simple graph with n vertices, (n should be grater than or equal to 3) euch that degree of every vertex is at least n/2, then G has a Hamilton Circuit.
\end{itemize}

\subproblem{a} \solution\\
This Graph has 17 vertices. But degree of every vertices is not greater than or equal to 17/2=8.5.
So this Graph does not have a Hamilton Circuit.

\begin{figure*}[htb]
	\centering
	\includegraphics[scale=0.3]{G1.png}
	\caption{Graph G1}
	\label{fig:graph1}
\end{figure*}

\subproblem{b} \solution\\

\begin{figure*}[htb]
	\centering
	\includegraphics[scale=0.3]{G2.png}
	\caption{Graph G1}
	\label{fig:graph2}
\end{figure*}

In this Graph we have a vertex that have just 1 edge. So this Graph cannot have a Hamilton Circuit.
\newpage
\subproblem{c} \solution\\

\begin{figure*}[htb]
	\centering
	\includegraphics[scale=0.3]{G3.png}
	\caption{Graph G1}
	\label{fig:graph2}
\end{figure*}

\problem{3: Applications on Graphs}{20}
\solution
\begin{figure*}[htb]
	\centering
	\includegraphics[scale=0.2]{GrapLesson.png}
	\caption{Solution}
	\label{fig:graph2}
\end{figure*}
\newline

In the Graph, I represented the vertices as the lessons which are not took together by students.
\newline
So, I designed the time slots like in the figure.
MATH101 , CSE211 and MATH243 are not took together, we can do these exams at the same time.
It can be applied to CSE346 and CSE102 also.





\end{document}