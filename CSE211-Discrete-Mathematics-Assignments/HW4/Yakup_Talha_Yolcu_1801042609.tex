\documentclass[a4 paper]{article}
\usepackage[inner=2.0cm,outer=2.0cm,top=2.5cm,bottom=2.5cm]{geometry}
\usepackage{setspace}
\usepackage[rgb]{xcolor}
\usepackage{verbatim}
\usepackage{subcaption}
\usepackage{amsgen,amsmath,amstext,amsbsy,amsopn,tikz,amssymb}
\usepackage[colorlinks=true, urlcolor=blue,  linkcolor=blue, citecolor=blue]{hyperref}
\usepackage[colorinlistoftodos]{todonotes}
\usepackage{rotating}
\usepackage{booktabs}
\newcommand{\ra}[1]{\renewcommand{\arraystretch}{#1}}

\newtheorem{thm}{Theorem}[section]
\newtheorem{prop}[thm]{Proposition}
\newtheorem{lem}[thm]{Lemma}
\newtheorem{cor}[thm]{Corollary}
\newtheorem{defn}[thm]{Definition}
\newtheorem{rem}[thm]{Remark}
\numberwithin{equation}{section}

\newcommand{\homework}[6]{
   \pagestyle{myheadings}
   \thispagestyle{plain}
   \newpage
   \setcounter{page}{1}
   \noindent
   \begin{center}
   \framebox{
      \vbox{\vspace{2mm}
    \hbox to 6.28in { {\bf CSE 211:~Discrete Mathematics \hfill {\small (#2)}} }
       \vspace{6mm}
       \hbox to 6.28in { {\Large \hfill #1  \hfill} }
       \vspace{6mm}
       \hbox to 6.28in { {\it Instructor: {\rm #3} \hfill Name: {\rm #5} \hfill Student Id: {\rm #6}} \hfill}
       \hbox to 6.28in { {\it Assistant: #4  \hfill #6}}
      \vspace{2mm}}
   }
   \end{center}
   \markboth{#5 -- #1}{#5 -- #1}
   \vspace*{4mm}
}

\newcommand{\problem}[2]{~\\\fbox{\textbf{Problem #1}}\hfill (#2 points)\newline\newline}
\newcommand{\subproblem}[1]{~\newline\textbf{(#1)}}
\newcommand{\D}{\mathcal{D}}
\newcommand{\Hy}{\mathcal{H}}
\newcommand{\VS}{\textrm{VS}}
\newcommand{\solution}{~\newline\textbf{\textit{(Solution)}} }

\newcommand{\bbF}{\mathbb{F}}
\newcommand{\bbX}{\mathbb{X}}
\newcommand{\bI}{\mathbf{I}}
\newcommand{\bX}{\mathbf{X}}
\newcommand{\bY}{\mathbf{Y}}
\newcommand{\bepsilon}{\boldsymbol{\epsilon}}
\newcommand{\balpha}{\boldsymbol{\alpha}}
\newcommand{\bbeta}{\boldsymbol{\beta}}
\newcommand{\0}{\mathbf{0}}


\begin{document}
\homework{Homework \#4}{Due: 17/01/21}{Dr. Zafeirakis Zafeirakopoulos}{Gizem S\"ung\"u}{Yakup Talha Yolcu}{1801042609}

\problem{1}{15+15=30}
Consider the nonhomogeneous linear recurrence relation $a_n$ = 3$a_{n-1}$ + $2^n$ .\\
\subproblem{a} Show that whether $a_n$ = $-2^{n+1}$ is a solution of the given recurrence relation or not. Show your work step by step.
\solution
\newline
If we replace $a_n$ = $-2^{n+1}$ we will get : \newline
$-2^{n+1}$= 3 ($-2^n$) + $2^n$ \newline
$-2^{n+1}$ = $-2^{n+1}$ we can't have a solution by this equation

\newline
\subproblem{b} Find the solution with $a_0$ = 1.
\solution
\newline
if we replace $a_n$ = x \newline
characteristic equation = x-3 roots is x=3\newline
my particular guess is A$2^n$ if I replace $a_n$ = A$2^n$ \newline
 A$2^n$=3 A$2^n-1$ + $2^n$ \newline
 We have A=-2 
 \newline
 so $a_n$ = $\alpha3^n$ - 2$2^n$ 
 \newline If we try $a_0$ = 1 we will get: \newline
 1=$\alpha$-2 so $\alpha$=3 \newline
 $a_n$ = 3 $3^n$ - 2 $2^n$
\newline



\problem{2}{35}
Solve the recurrence relation f(n) = 4f(n-1) - 4f(n-2) + $n^2$ for f(0) = 2 and f(1) = 5. 
\solution
\newline
$a_n$ = 4 $a_{n-1}$ - 4 $a_{n-2}$ + $n^2$ for $a_0$ = 2 and $a_1$ = 5 \newline
Characteristic equation: $x^2$ - 4x + 4 = 0 \newline
Roots are 2 and 2 so homogenius $a_n$ = $\alpha$$2^n$ + $\beta$$n2^n$ \newline
Particular guess is : $a_n$ = $A_2n^2$ + $A_1n$ + $A_0$ \newline
So $n^2$ = $A_2n^2$ + $A_1n$ + $A_0$ - 4 ($A_2(n-1)^2$ + $A_1(n-1)$ + $A_0$) + 4 ($A_2(n-2)^2$ + $A_1(n-2)$ + $A_0$) \newline
$n^2$ = $A_2n^2$ + $A_1n$ + $A_0$ - 4$A_2n^2$ + 8$A_2n$ - 4$A_2$ - 4 $A_1n$ + 4$A_1$ - 4$A_0$ + 4$A_2n^2$ - 16 $A_2n$ + 16$A_2$ + 4 $A_1n$ - 8$A_1$ + 4$A_0$ \newline
If we compare the $n^2$ 's coefficients  1=$A_2$ - 4$A_2$ + 4$A_2$ so $A_2$=1 \newline
If we compare n's coefficients 0 = $A_1$+8$A_2$-4$A_1$-16$A_2$+4$A_1$\newline
0=$A_1$+8-4$A_1$-16+4$A_1$\newline
8=$A_1$ \newline If we compare the constants coefficients:
0=$A_0$-4$A_2$-4$A_0$+16$A_2$-8$A_1$+4$A_0$ \newline
0=$A_0$-4-4$A_0$+16-64+4$A_0$
52=$A_0$

so our guess is become :  $a_n$ = $n^2$ + 8n + 52
\newline
$a_n$ = $\alpha$$2^n$ + $\beta$$n2^n$ + $n^2$ + 8n + 52 \newline
$\alpha$= -50 and $\beta$ = 22 so $a_n$ = (-50)$2^n$ + 22$n2^n$ + $n^2$ + 8n + 52 \newline



\problem{3}{20+15 = 35}
Consider the linear homogeneous recurrence relation $a_n$ = 2$a_{n-1}$ - 2$a_{n-2}$.
\subproblem{a} Find the characteristic roots of the recurrence relation.
\solution
characteristic equation is: 
$x^2$ - 2x + 2 = 0 \newline
$\Delta$ = $b^2$ - 4ac=-4<0 \newline
roots are : 1 - i and 1 + i
so we will have : \newline
$a_n$ = $\alpha (1-i)^n$ + $\beta(1+i)^n$

\newline
\subproblem{b} Find the solution of the recurrence relation with $a_0$ = 1 and $a_1$ = 2.
\solution
$a_n$ = $\alpha (1-i)^n$ + $\beta(1+i)^n$
\newline
if we try for $a_0$=$\alpha$ + $\beta$ = 1\newline
if we try for  $a_1$=$\alpha(1-i)$ + $\beta(1+i)$ = 2 \newline
if we replace $\alpha$=1-$\beta$
1-$\beta$ (1-i) + $\beta$(1+i)=2 \newline
2$\beta$i=1+i  multiply both sides with i \newline
-2$\beta$ = i - 1 so $\beta$ = (1-i)/2 \newline
$\alpha$=1-(1-i)/2 = (1 + i)/2
\newline
so $a_n$ = $\frac{1+i}{2} (1-i)^n$ + $\frac{1-i}{2}(1+i)^n$



\end{document} 


